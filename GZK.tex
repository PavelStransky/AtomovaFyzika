\section{Práh reakce}
\subsection{Greisenův-Zatsepinův-Kuzminův limit}
    GZK limit je prahová hodnota energie kosmického protonového záření, nad kterou dojde k interakci protonu s fotonem reliktního záření za vzniku buď protonu a neutrálního pionu, nebo neutronu a kladně nabitého pionu,\footnote{
        K reakci dochází přes $\Delta^{+}$ rezonanci.
    }
    \begin{subequations}
        \begin{align}
            p^{+}+\gamma_{\mathrm{RZ}}
                &\longrightarrow p^{+}+\pi^{0},\\
                &\longrightarrow n^{0}+\pi^{+},
        \end{align}        
    \end{subequations}
    čímž vysokonenergetický foton ztratí energii (je zbržděn).
    Odbvoďte tento limit pro obě reakce.

\subsection{Vznik elektron-pozitronového páru}
    Určete, jakou energii musí mít foton, aby mohl způsobit vznik páru elektron-positron.
    Proč k uvedené reakci nejsnáze dochází v pevných látkách?
    Je možné, aby elektron-pozitronový pár vznikl při lékařském rentgenu?

\subsection{Fotoefekt}
    Vysvětlete, proč nemůže foton odevzdat veškerou svou energii a hybnost volnému elektronu.
    K fotoelektrickému jevu tedy může dojít jen tehdy, když foton zainteraguje s vázaným elektronem.