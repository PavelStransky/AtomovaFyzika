\section{Víceelektronové atomy}
    Dokažte Unsöldův teorém, který říká, že pro dané orbitální kvantové číslo $l$ je součet hustot pravděpodobnosti pro všechny stavy s magnetickým kvantovým číslem $m_l=-l,\dotsc,l$ nezávislý na úhlech $\theta,\phi$.
    Teorém dokažte pro $l=0$, $l=1$, $l=2$.

\subsection{Mnohaelektronový atom}
    Na základě jednoduchého Bohrova modelu atomu odhadněte rozměr atomu s protonovým číslem~$Z$.

    \begin{enumerate}
    \item
        Jaký je poloměr kruhové dráhy elektronu v Bohrově modelu, pokud jádro nese náboj $Ze$ a hladina má kvantové číslo $n$?
        Vyjádřete v násobcích Bohrova poloměru $a_{0}$ pro atom vodíku.

    \item
        Předpokládejte, atomové jádro s nábojem $Ze$ doplníte $Z$ elektrony, které mezi sebou navzájem neinteragují, a zanedbejte i spin-orbitální vazbu a další případné interakce.
        Jaký bude poloměr výsledného atomu v případě, že je ve valenční slupce jen jeden elektron, a v případě, že je valenční slupka zcela zaplněna?
        Budou atomy větší, nebo menší v porovnání s atomem vodíku?
    \end{enumerate}

    K výpočtu použijte vzorec
    \begin{equation}
        \sum_{n=1}^{N}n^{2}=\frac{N}{6}(N+1)(2N+1).
    \end{equation}