\section{Víceelektronové atomy}
\subsection{Unsöldův teorém}
    Dokažte Unsöldův teorém, který říká, že pro dané orbitální kvantové číslo $l$ je součet hustot pravděpodobnosti pro všechny stavy s magnetickým kvantovým číslem $m_l=-l,\dotsc,l$ nezávislý na úhlech $\theta,\phi$.
    Teorém dokažte pro $l=0$, $l=1$, $l=2$.

\subsection{Mnohaelektronový atom}
    Na základě Bohrova modelu atomu odhadněte rozměr atomu s protonovým číslem~$Z$.

    \begin{enumerate}
        \item
            Jaký je poloměr kruhové dráhy elektronu v Bohrově modelu, pokud jádro nese náboj $Ze$ a hladina má kvantové číslo $n$?
            Vyjádřete v násobcích Bohrova poloměru $a_{0}$ pro atom vodíku.

        \item
            Spočítejte poloměr atomu vzácných plynů (zaplněná valenční slupka) a alkalických kovů (jeden elektron ve valenční slupce) a porovnejte s Bohrovým poloměrem $a_{0}$.
    \end{enumerate}

