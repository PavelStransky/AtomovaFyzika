\section{Kvantové provázání}
    \subsection{Kvantová teleportace}
        Popište, jakým způsobem lze přenést stav kvantového qubitu z místa A do místa B pomocí klasického komunikačního kanálu. Tento jev se nazývá kvantová teleportace.

    \subsection{Provázaný stav}
        Dokažte, že provázaný stav dvou qubitů 
        \begin{equation}
            \ket{\Psi}=\frac{1}{\sqrt{2}}\left(\ket{\uparrow}_{1}\ket{\uparrow}_{2}+\ket{\downarrow}_{1}\ket{\downarrow}_{2}\right)
        \end{equation}
        nelze faktorizovat, tj. nelze ho napsat ve tvaru
        \begin{equation}
            \left(\alpha\ket{\uparrow}+\beta\ket{\downarrow}\right)_{1}\left(\gamma\ket{\uparrow}+\delta\ket{\downarrow}\right)_{2},\quad\alpha,\beta,\gamma,\delta\in\mathbb{C}.
        \end{equation}

    \subsection{Stav dvouqubitového systému}
        Rozhodněte, zda stav dvou qubitů
        \begin{equation}
    \ket{\Phi}=\mathcal{N}\left(2\ket{\uparrow\uparrow}+\im\ket{\downarrow\uparrow}+4\im\ket{\uparrow\downarrow}-2\ket{\downarrow\downarrow}\right)
        \end{equation}
        je provázaný, a svou odpověď zdůvodněte. 
        Nalezněte normalizační faktor $\mathcal{N}$.

        V zápisu stavu je použito zjednodušeného značení $\ket{\uparrow\downarrow}\equiv\ket{\uparrow}_{1}\otimes\ket{\downarrow}_{2}$ (analogicky i pro ostatní kombinace spinu nahoru a dolů).
