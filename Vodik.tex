\section{Atom vodíku}
\subsection{Klasický atom vodíku}
    Spočítejte frekvenci kruhového pohybu elektronu v klasickém modelu vodíkového atomu (elektron se nachází ve vzdálenosti Bohrova poloměru $a_{0}$ od atomového jádra).
    Pokud by elektron vyzařoval elektromagnetické záření o této frekvenci, v jaké oblasti spektra by se nacházelo?

\subsection{Nestabilita klasického atomu}
    Nabitá částice s nábojem $q$ pohybující se se zrychlením $a$ vyzařuje podle klasické teorie elektromagnetického záření s výkonem
    \begin{equation}
        P=\frac{2}{3}\frac{q^2}{4\pi\epsilon}\frac{1}{c^3}a^{2},
    \end{equation}
    kde $\epsilon$ je permitivita a $c$ je rychlost světla.\footnote{Vztah lze najít pod názvem \emph{Larmorův vztah}. Záření se nazývá podle charakteru zrychlení \emph{synchrotronové záření} nebo \emph{brzdné záření}.}

    \begin{itemize}
        \item
            Pokud by se elektron v atomu vodíku choval jako klasická nabitá částice, spočítejte, za jak dlouho by dopadl na jádro z kruhové dráhy o poloměru daném Bohrovým poloměrem.
        \item
            Určete průměrný vyzařovaný výkon.
    \end{itemize}

\subsection{Bohrův model atomu}
    Odvoďte možné hodnoty energií elektronu atomu vodíku za Bohrova předpokladu, že elektron obíhá okolo atomového jádra díky elektrostatické síle mezi ním a protonem v jádře a že pokud jeho moment hybnosti nabývá celočíselných násobků redukované Planckovy konstanty $\hbar$, nedochází ke zrátě energie Larmorovým vyzařováním.

    Jak se změní výsledek, pokud bude mít jádro náboj $Ze$, $Z>1$?

\subsection{Vlnové délky spektrálních čar atomu vodíku}
    Odvoďte nejkratší a nejdelší vlnovou délku pro Lymanovu, Balmerovu a Paschenovu sérii spektrálních čar atomu vodíku. Které z těchto čar budou ležet ve viditelném světle?

\subsection{Energie fotonů viditelného světla}
    Určete rozmezí energií fotonů viditelného světla.

\subsection{Makroskopický atom}
    Pro jak velké hlavní kvantové číslo bude mít atom vodíku rozměr $r=1\unit{cm}$?

\subsection{Degenerace hladin atomu vodíku}
    Určete stupeň degenerace (počet různých kombinací kvantových čísel, kterými lze získat danou energetickou hladinu) hladiny atomu vodíku s hlavním kvantovým číslem $n$.

\subsection{Poloměr atomu vodíku}
    Ze znalosti radiální části vlnové funkce základního stavu atomu vodíku určete vzdálenost od středu jádra, na které najdete elektron nejpravděpodobněji, a střední poloměr atomu (střední hodnotu polohy).
