\documentclass[a4paper,11pt,twoside]{article}

\usepackage{palatino,newtxmath,bbold}	%% Písma
\usepackage{microtype}					%% Lepší mezery
\usepackage[T1]{fontenc}
\usepackage[utf8]{inputenc}	            %% Kódování textu
\usepackage[czech]{babel}               %% České nápisy
\usepackage{amsfonts,amsmath}			%% Matematické symboly (amssymb koliduje s jiným balíkem)

\usepackage{epsfig}                     %% Obrázky
\usepackage[subrefformat=simple,labelformat=simple]{subcaption} %% Podobrázky
\usepackage{graphicx}					%% Doplňující příkazy pro obrázky

\usepackage{xifthen}					%% Podmínka if - then
\usepackage{makeidx}					%% Rejstřík
%\usepackage{showframe}
%\usepackage{showidx}

\usepackage[multiple]{footmisc}			%% Lepší formátování poznámek pod čarou - nefunguje s hyperref
%\usepackage{fnpct}						%% Lepší formátování poznámek pod čarou
\usepackage{comment}					%% Komentáře
\usepackage{scrextend}					%% Vylepšené formátování (addmargin)
\usepackage{xcolor}						%% Barvy
\usepackage{indentfirst}				%% Odsazení prvního odstavce
\usepackage{fancyhdr}
\usepackage{blkarray}					%% Pro komentované vektory
\usepackage{empheq}                     %% Box around equations

\usepackage{csquotes}
\usepackage{expl3}						%% Jinak nefunguje biblatex
\usepackage{biblatex}
\addbibresource{S:/Fyzika/Bibliography/References.bib}

%\bibliographystyle{phaip}

\graphicspath{{figures/}}

%\usepackage{showlabels}                 %% Temporarily show the names of labels
%\renewcommand{\showlabelfont}{\tiny\bfseries\color{black}}

\usepackage{mathtools}
%\mathtoolsset{showonlyrefs}            %% Remove equation number of unrefferenced equations

\usepackage[unicode]{hyperref}			%% Hypertextové odkazy
\hypersetup{
	pdftitle={Cvičení k přednášce Atomová fyzika (NFUF301)},
	pdfauthor={Pavel Stránský},
	pdffitwindow=true,
	colorlinks=true,
	urlcolor=cyan,            			%barva textu pri tisku
	linkcolor=red,
	citecolor=green,
	filecolor=magenta
}

% Velikost stránky
\addtolength{\topmargin}{-1.5cm} %\addtolength{\textheight}{-10cm}
\addtolength{\textwidth}{4cm} \addtolength{\textheight}{4cm} % Šířka a výška textu
\addtolength{\voffset}{-0.5cm} % Horní okraj
\addtolength{\hoffset}{-2cm}
\setlength{\headheight}{15pt}

\pagestyle{fancy}

\input{defs}

\includecomment{theory}
%\excludecomment{solution}
%\includecomment{note}

% Homework - části, které jsou (byly) za domácí úkol, a proto by se neměly vyskytnout ve sbírce
%\excludecomment{homework}

% homeworknote - části, které jsou navázané na řešení; část s domácím úkolem; vzájemně exklusivní s prostředním homework
\newenvironment{homeworknote}{}{}

% Toto se odkomentuje pro tištěnou kompletní sbírku
%\excludecomment{homeworknote}
\newenvironment{homework}{}{}

% solution - část s řešením
%\excludecomment{solution}
%\excludecomment{theory}
\newenvironment{solution}{\begin{addmargin}{0.5cm}\color{gray}\subsubsection*{Řešení:}\small}{\end{addmargin}\vspace*{0.3cm}}

\newenvironment{example}{\textbf{\textit{Příklad:}}}{}

\newenvironment{note}[1][]{\vspace*{0.2cm}\noindent\textbf{\textit{\ifthenelse{\isempty{#1}}{Poznámka: }{#1}}}}{}
%\newenvironment{theory}{}{}

\def\sec#1{\subsubsection*{#1}}
\def\sfootnote#1{\footnote{\color{gray}#1}}
\def\scaption#1{\caption{\small\color{gray}#1}}
\def\scaptionx#1#2{\caption[#1]{\small\color{gray}#2}}

\newcommand{\np}{\clearpage\newpage}
%\newcommand{\np}{\clearpage\setcounter{page}{1}\newpage}
%\newcommand{\np}{}\newcommand{\minput}[1]{\input{#1}}

\newcommand{\exercise}[2][]{\ifthenelse{\isempty{#1}}
	{\np\thispagestyle{empty}\subsubsection*{Domácí úkol -- #2}}
	{\np\thispagestyle{empty}\np\subsubsection*{Domácí úkol -- #2 \small{\it{(termín odevzdání: {#1})}}}}
}

\makeindex

\begin{document}

\makeatletter
\@addtoreset{equation}{subsection}
\renewcommand{\theequation}{\arabic{section}.\arabic{subsection}.\arabic{equation}}
%\renewcommand{\thepage}{\arabic{section}.\arabic{page}}
%\@addtoreset{page}{section}
\makeatother

\title{Cvičení k přednášce Atomová fyzika (NFUF301)}
\date{\today}
\author{Pavel Stránský}

\maketitle
\pdfbookmark{\contentsname}{Contents}
\tableofcontents\np

\section{Černé těleso}
\subsection{Rayleighův-Jeansův zákon}
    Odvoďte objemovou hustotu energie černého tělesa pro frekvenci $\nu$ a vlnovou délku $\lambda$.
    Předpokládejte, že energie jednotlivých módů elektromagnetického záření může nabývat jakýchkoliv hodnot.

\subsection{Planckův zákon}
    Odvoďte objemovou hustotu energie černého tělesa za předpokladu, že energie jednotlivých energie módů elektromagnetického záření může nabývat jen celočíselných násobků frekvence módů $\nu$,\footnote{
        Vztah lze ekvivalentně zapsat pomocí úhlové frekvence $\omega$ a redukované Planckovy konstanty $\hbar$ jako
        \begin{equation}
            E_{n}=\hbar\omega n
        \end{equation}
    }
    \begin{equation*}
        E_{n}=h\nu n,
    \end{equation*}
    kde $n$ je přirozené číslo a $h$ je konstanta (Planckova konstanta).

\subsection{Wienův posunovací zákon}
    Odvoďte, pro jakou frekvenci a pro jakou vlnovou délku je objemová hustota energie černého tělesa daná Planckovým zákonem maximální.

\subsection{Stefanův-Boltzmannův zákon}
    Odvoďte celkový zářivý výkon černého tělesa o teplotě $T$.

\subsection{Střední energie fotonu}
    Určete počet fotonů v jednotkovém objemu pro frekvenci $\nu$ a vlnovou délku $\lambda$ a celkový počet fotonů přes všechny vlnové délky.
    Jaká je střední energie jednoho fotonu v záření černého tělesa o teplotě $T$?

\subsection{Teplota Slunce}
    Je-li Slunce v zenitu, je intenzita slunečního záření dopadající na vodorovný zemský povrch $I_{\oplus}=1367\unit{Wm^{-2}}$.
    Za předpokladu, že vyzařování Slunce lze považovat za záření černého tělesa, a znáte-li poloměr Slunce $R_{\odot}$ a vzdálenost Země od Slunce $d$, určete teplotu na povchu Slunce.

\subsection{Ztráta hmotnosti Slunce}
    Jakou hmotnost ztratí Slunce vyzařováním za $1\unit{s}$?

\subsection{Žárovka}
    Wolframové vlákno v klasické žárovce se rozžhaví na teplotu $T=3000\unit{K}$.
    Jaké procento vyzařované energie je ve viditelné části spektra mezi vlnovými délkami $\lambda\in[380\unit{nm},750\unit{nm}]$?

\subsection{Hlava}
    Odhadněte celkový zářivý výkon holé lidské hlavy bez pokrývky.
    Jaký je rozdíl zářivého výkonu a zářivého příkonu v prostředí, které má $t_{\text{okolí}}=0\unit{^\circ C}$?
    Bazální metabolismus dospělého člověka je přibližně $P_{B}=1700\unit{kcal\,den^{-1}}$.
    Určete, jaké procento energie získané metabolismem se v chladném počasí ztratí hlavou pouhým vyzařováním.\footnote{
        Proto je dobré nosit v zimě čepici.
    }

\subsection{Fotonová plachetnice}
    Určete, jaká síla by díky slunečnímu záření působila na čtvercovou plachtu o rozměru $100\unit{m}\times 100\unit{m}$, nacházející se na oběžné dráze Země. 
    Jak musí být plachta orientovaná, aby síla byla co největší?
    Je síla větší, když plachta záření pohltí, nebo když ho odrazí?

\subsection{Vlákno žárovky}
    Odhadněte délku a poloměr wolframového vlákna žárovky s příkonem $P=100\unit{W}$, víte-li, že teplota vlákna je $T=2700\unit{K}$.

\subsection{Kosmické mikrovlnné záření}
    Kosmické mikrovlnné záření (reliktní záření) je odkaz z počátečních fází vývoje vesmíru.
    Má charakter přibližně izotropního záření černého tělesa o teplotě $T\approx2\c7\unit{K}$.
    Určete, na jaké frekvenci a pro jakou vlnovou délku je hustota energie nejvyšší.
    Spočítejte, kolik fotonů reliktního záření dopadá na jednotkovou plochu zemského povrchu za sekundu.\np
\section{Částicový charakter elektromagnetického záření}
    \subsection{Comptonův rozptyl}
        Odvoďte vztah pro energii fotonu $E'_{\gamma}$ a jeho vlnovou délku $\lambda'$, který se rozptýlil na elektronu na úhel $\theta$ (obrázek~\ref{fig:Compton}).
        Energie a vlnová délka fotonu před rozptylem je $E_{\gamma}$ a $\lambda$.
        Předpokládejte, že před rozptylem je elektron v klidu.
        Hmotnost elektronu je $m_{e}$.

        \begin{figure}[!h]
            \centering
            \includegraphics[width=0.6\linewidth]{Compton.png}
            \caption{Comptonův rozptyl fotonu $\gamma$ na elektronu $e^{-}$.}
            \label{fig:Compton}
        \end{figure}

    \subsection{Comptonova vlnová délka}
        Vyjádřete vztah pro změnu vlnové délky fotonu při Comptonově rozptylu $\Delta\lambda=\lambda'-\lambda$ pomocí Comptonovy vlnové délky $\lambda_{c}=h/(m_{e}c)$, kde $h$ je Planckova konstanta, $m_{e}$ hmotnost elektronu a $c$ rychlost světla.

    \subsection{Úhel vylétávajícího elektronu}
        Odvoďte vztah pro úhel $\varphi$, pod kterým vylétá elektron po Comptonově rozptylu (obrázek~\ref{fig:Compton}).

    \subsection{Spektrum $\gamma$ při měření v detektoru}
        \begin{figure}[!h]
            \centering
            \includegraphics[width=0.35\linewidth]{ComptonSpectrum.png}
            \caption{Detekované Comptonovské spektrum monochromatického $\gamma$ záření.}
            \label{fig:ComptonSpectrum}
        \end{figure}        
    
        Detektor vysokoenergetických kvant $\gamma$ funguje na principu Comptonova rozptylu, kdy kinetická energie rozptýlených elektronů vytvoří impuls elektrického proudu, který se následně zesílí a změří.

        Předpokládejte, že na detektor dopadá monochromatické záření s energií kvant $E_{\gamma}$ vzniklé rozpadem radioaktivního nuklidu.
        Vysvětlete body a, b, c z obrázku~\ref{fig:ComptonSpectrum}; obrázek zobrazuje četnost, se kterou byla detektorem naměřena energie elektronu $E$. 
        Odhadněte, jaká byla energie $E_{\gamma}$, a z této \href{https://cds.cern.ch/record/1309915/files/978-3-642-02586-0_BookBackMatter.pdf}{tabulky} určete nuklid, jehož rozpad je měřen.

        % Zdroj tabulky: https://www.ld-didactic.de/software/524221en/Content/Appendix/ComptonSpectrum.htm\np
\section{Práh reakce}
\subsection{Greisenův-Zatsepinův-Kuzminův limit}
    GZK limit je prahová hodnota energie kosmického protonového záření, nad kterou dojde k interakci protonu s fotonem reliktního záření za vzniku buď protonu a neutrálního pionu, nebo neutronu a kladně nabitého pionu:\footnote{
        K reakci dochází přes $\Delta^{+}$ rezonanci.
    }
    \begin{subequations}
        \begin{align}
            p^{+}+\gamma_{\mathrm{RZ}}
                &\longrightarrow p^{+}+\pi^{0},\\
                &\longrightarrow n^{0}+\pi^{+}.
        \end{align}        
    \end{subequations}
    Odbvoďte tento limit pro obě reakce.
\np
\section{Rozptyl}
\subsection{Srážkový parametr a diferenciální účinný průřez}
    Odvoďte vztah mezi srážkovým parametrem $b(\theta)$ a diferenciálním účinným průřezem $\derivative{\sigma}{\Omega}$.

\subsection{Rutherfordův rozptyl}
    Vztah pro srážkový parametr u Rutherfordova rozptylu (rozptyl $\alpha$ částice na jádru s protonovým číslem $Z$) na úhel $\theta$ je
    \begin{equation}
        b(\theta)=\frac{d_{0}}{2}\frac{1}{\tg{\frac{\theta}{2}}},
    \end{equation}
    kde
    \begin{equation}
        d_{0}=\frac{2Ze^{2}}{4\pi\epsilon_{0}}\frac{1}{T}
    \end{equation}
    je~\emph{Sommerfeldův parametr} (vzdálenost nejbližšího přiblížení rozptylující a rozptylované částice) a $T$ je kinetická energie $\alpha$ částice v laboratorní soustavě.

    Odvoďte výraz pro diferenciální účinný průřez.

\subsection{Rozptyl na tvrdé kouli}
    \begin{enumerate}
        \item Odvoďte vztah mezi srážkovým parametrem a rozptylem na úhel $\theta$ pro tvrdou kouli.
        \item Odvoďte výraz pro diferenciální účinný průřez.
        \item Určete celkový účinný průřez.
    \end{enumerate}\np
\section{Atom vodíku}
\subsection{Nestabilita klasického atomu}
    Nabitá částice s nábojem $q$ pohybující se se zrychlením $a$ vyzařuje podle klasické teorie elektromagnetického záření s výkonem
    \begin{equation}
        P=\frac{2}{3}\frac{q^2}{4\pi\epsilon}\frac{1}{c^3}a^{2},
    \end{equation}
    kde $\epsilon$ je permitivita a $c$ je rychlost světla.
    Spočítejte, za jak dlouho by dopadl elektron atomu vodíku na jádro, kdyby se pohyboval jako klasická nabitá částice z kruhové dráhy o poloměru daném Bohrovým poloměrem.

    Určete průměrný vyzařovaný výkon.

\subsection{Bohrův model atomu}
    Odvoďte možné hodnoty energií elektronu atomu vodíku za Bohrova předpokladu, že elektron obíhá okolo atomového jádra a že pokud jeho moment hybnosti nabývá celočíselných násobků redukované Planckovy konstanty $\hbar$, nedochází ke zrátě energie Larmorovým vyzařováním.

    Jak se změní výsledek, pokud bude mít jádro náboj $Ze$, $Z>1$?

\subsection{Vlnové délky spektrálních čar atomu vodíku}
    Odvoďte nejkratší a nejdelší vlnovou délku pro Lymanovu, Balmerovu a Paschenovu sérii spektrálních čar atomu vodíku. Které z těchto čar budou ležet ve viditelném světle?

\subsection{Energie fotonů viditelného světla}
    Určete rozmezí energií fotonů viditelného světla.

\subsection{Makroskopický atom}
    Pro jak velké hlavní kvantové číslo bude mít atom vodíku rozměr $r=1\unit{cm}$?

\subsection{Degenerace hladin atomu vodíku}
    Určete stupeň degenerace (počet různých kombinací kvantových čísel, kterými lze získat danou energetickou hladinu) hladiny atomu vodíku s hlavním kvantovým číslem $n$.

\subsection{Poloměr atomu vodíku}
    Ze znalosti radiální části vlnové funkce základního stavu atomu vodíku určete střední poloměr atomu.

\subsection{Mnohaelektronový atom}
    Na základě jednoduchého Bohrova modelu atomu odhadněte rozměr atomu s protonovým číslem~$Z$.

    \begin{enumerate}
    \item
        Jaký je poloměr kruhové dráhy elektronu v Bohrově modelu, pokud jádro nese náboj $Ze$ a hladina má kvantové číslo $n$?
        Vyjádřete v násobcích Bohrova poloměru $a_{0}$ pro atom vodíku.

    \item
        Předpokládejte, atomové jádro s nábojem $Ze$ doplníte $Z$ elektrony, které mezi sebou navzájem neinteragují, a zanedbejte i spin-orbitální vazbu a další případné interakce.
        Jaký bude poloměr výsledného atomu v případě, že je ve valenční slupce jen jeden elektron, a v případě, že je valenční slupka zcela zaplněna?
        Budou atomy větší, nebo menší v porovnání s atomem vodíku?
    \end{enumerate}

    K výpočtu použijte vzorec
    \begin{equation}
        \sum_{n=1}^{N}n^{2}=\frac{N}{6}(N+1)(2N+1).
    \end{equation}\np

\printindex

%\bibliography{Z:/Share/Fyzika/Bibliography/References.bib}
\printbibliography
%\input{refs}

\end{document}
