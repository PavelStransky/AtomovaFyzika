\section{Molekuly a chemická vazba}
    \subsection{Výhřevnost uhlí odhadem}
        Odhadněte výhřevnost uhlí na základě hrubého odhadu, že při reakci hoření uhlí
        \begin{equation}
            \ce{C + O2 -> CO2}
        \end{equation}
        se vznikem jedné molekuly uvolní energie $4\unit{eV}$.
        Kolik váží oxid uhličitý vzniklý spálením $1\unit{kg}$ uhlí?

    \subsection{Výhřevnost uhlí přesně}
        Spočítejte výhřevnost uhlí přesně, pokud znáte slučovací enthalpii
        \begin{equation}
            \Delta H_{f}^{\Theta}(\ce{CO2})=-393\unit{kJ}\unit{mol^{-1}}.
        \end{equation}
        O kolik je důsledkem relativistického vztahu mezi hmotností a energií vzniklý oxid uhličitý lehčí než jsou hmotnosti konstituentů před reakcí?

    \subsection{Hustoty látek}
        Odhadněte rozmezí hustot plynů a pevných látek.

    \subsection{Odhadněte výšku hor}
        Předpokládejte, že hory jsou tvořeny křemenem (oxidem křemičitým \ce{SiO2}).
        Odhadněte maximální výšku hor na Zemi a na Marsu.\footnote{Úloha je inspirována článkem~\cite{Weisskopf1975}.}
        Odhadněte, jaká je minimální velikost vesmírného objektu, aby se začal zakulacovat.