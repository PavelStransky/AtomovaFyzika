\section{Černé těleso}
\subsection{Rayleighův-Jeansův zákon}
    Odvoďte objemovou hustotu energie černého tělesa pro frekvenci $\nu$ a vlnovou délku $\lambda$.
    Předpokládejte, že energie jednotlivých módů elektromagnetického záření může nabývat jakýchkoliv hodnot.

\subsection{Planckův zákon}
    Odvoďte objemovou hustotu energie černého tělesa za předpokladu, že energie jednotlivých energie módů elektromagnetického záření může nabývat jen celočíselných násobků frekvence módů $\nu$,\footnote{
        Vztah lze ekvivalentně zapsat pomocí úhlové frekvence $\omega$ a redukované Planckovy konstanty $\hbar$ jako
        \begin{equation}
            E_{n}=\hbar\omega n
        \end{equation}
    }
    \begin{equation*}
        E_{n}=h\nu n,
    \end{equation*}
    kde $n$ je přirozené číslo a $h$ je konstanta (Planckova konstanta).

\subsection{Wienův posunovací zákon}
    Odvoďte, pro jakou frekvenci a pro jakou vlnovou délku je objemová hustota energie černého tělesa daná Planckovým zákonem maximální.

\subsection{Stefanův-Boltzmannův zákon}
    Odvoďte celkový zářivý výkon černého tělesa o teplotě $T$.

\subsection{Střední energie fotonu}
    Určete počet fotonů v jednotkovém objemu pro frekvenci $\nu$ a vlnovou délku $\lambda$ a celkový počet fotonů přes všechny vlnové délky.
    Jaká je střední energie jednoho fotonu v záření černého tělesa o teplotě $T$?

\subsection{Teplota Slunce}
    Je-li Slunce v zenitu, je intenzita slunečního záření dopadající na vodorovný zemský povrch $I_{\oplus}=1367\unit{Wm^{-2}}$.
    Za předpokladu, že vyzařování Slunce lze považovat za záření černého tělesa, a znáte-li poloměr Slunce $R_{\odot}$ a vzdálenost Země od Slunce $d$, určete teplotu na povchu Slunce.

\subsection{Ztráta hmotnosti Slunce}
    Jakou hmotnost ztratí Slunce vyzařováním za $1\unit{s}$?

\subsection{Žárovka}
    Wolframové vlákno v klasické žárovce se rozžhaví na teplotu $T=3000\unit{K}$.
    Jaké procento vyzařované energie je ve viditelné části spektra mezi vlnovými délkami $\lambda\in[380\unit{nm},750\unit{nm}]$?

\subsection{Hlava}
    Odhadněte celkový zářivý výkon holé lidské hlavy bez pokrývky.
    Jaký je rozdíl zářivého výkonu a zářivého příkonu v prostředí, které má $t_{\text{okolí}}=0\unit{^\circ C}$?
    Bazální metabolismus dospělého člověka je přibližně $P_{B}=1700\unit{kcal\,den^{-1}}$.
    Určete, jaké procento energie získané metabolismem se v chladném počasí ztratí hlavou pouhým vyzařováním.\footnote{
        Proto je dobré nosit v zimě čepici.
    }

\subsection{Fotonová plachetnice}
    Určete, jaká síla by díky slunečnímu záření působila na čtvercovou plachtu o rozměru $100\unit{m}\times 100\unit{m}$, nacházející se na oběžné dráze Země. 
    Jak musí být plachta orientovaná, aby síla byla co největší?
    Je síla větší, když plachta záření pohltí, nebo když ho odrazí?

\subsection{Vlákno žárovky}
    Odhadněte délku a poloměr wolframového vlákna žárovky s příkonem $P=100\unit{W}$, víte-li, že teplota vlákna je $T=2700\unit{K}$.

\subsection{Kosmické mikrovlnné záření}
    Kosmické mikrovlnné záření (reliktní záření) je odkaz z počátečních fází vývoje vesmíru.
    Má charakter přibližně izotropního záření černého tělesa o teplotě $T\approx2\c7\unit{K}$.
    Určete, na jaké frekvenci a pro jakou vlnovou délku je hustota energie nejvyšší.
    Spočítejte, kolik fotonů reliktního záření dopadá na jednotkovou plochu zemského povrchu za sekundu.