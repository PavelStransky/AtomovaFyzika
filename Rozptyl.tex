\section{Rozptyl}
\subsection{Srážkový parametr a diferenciální účinný průřez}
    Odvoďte vztah mezi srážkovým parametrem $b(\theta)$ a diferenciálním účinným průřezem $\derivative{\sigma}{\Omega}$.

\subsection{Rutherfordův rozptyl}
    Vztah pro srážkový parametr u Rutherfordova rozptylu (rozptyl $\alpha$ částice na jádru s protonovým číslem $Z$) na úhel $\theta$ je
    \begin{equation}
        b(\theta)=\frac{d_{0}}{2}\frac{1}{\tg{\frac{\theta}{2}}},
    \end{equation}
    kde
    \begin{equation}
        d_{0}=\frac{2Ze^{2}}{4\pi\epsilon_{0}}\frac{1}{T}
    \end{equation}
    je~\emph{Sommerfeldův parametr} (vzdálenost nejbližšího přiblížení rozptylující a rozptylované částice) a $T$ je kinetická energie $\alpha$ částice v laboratorní soustavě.

    Odvoďte výraz pro diferenciální účinný průřez.

\subsection{Rozptyl na tvrdé kouli}
    \begin{enumerate}
        \item Odvoďte vztah mezi srážkovým parametrem a rozptylem na úhel $\theta$ pro tvrdou kouli.
        \item Odvoďte výraz pro diferenciální účinný průřez.
        \item Určete celkový účinný průřez.
    \end{enumerate}