\section{Nerozlišitelné částice}
    \subsection{Kvantový tlak}
        Určete střední energii jedné částice nerelativistického fermionového plynu o hustotě počtu částic $\rho$.
        Určete, jaký je v tomto plynu tlak za předpokladu, že termodynamickou teplotu lze zanedbat.

    \subsection{Relativistický kvantový tlak}
        Zopakujte řešení předchozí úlohy pro ultrarelativistický fermionový plyn (rychlosti částic plynu jsou tak velké, že lze zanedbat jejich klidovou hmotnost).

    \subsection{Poloměr hvězdy}
        Odhadněte poloměr vyhořelé hvězdy (bílého trpaslíka) s hmotností $M$.
        Předpokládejte, že hvězda je složena z uhlíků $^{12}$C a je homogenní (její hustota je konstantní, nezávislá na vzdálenosti od středu hvězdy).
        Spočítejte číselně pro $M=M_{\odot}$ (poloměr vyhořelého Slunce).

    \subsection{Chandrasekharova mez}
        Odhadněte Chandrasekharovu mez pro bílého trpaslíka (jedná se o mezní hmotnost, nad kterou již kvantový tlak elektronového plynu neudrží hvězdu proti gravitační síle a hvězda se zhroutí do neutronové hvězdy nebo černé díry).

    \subsection{Interakce způsobená nerozlišitelností volných částic}
        Uvažujte dvě nerozlišitelné volné částice o hmotnosti $m$ pohybující se na přímce.
        Jejich vlnové funkce jsou dány gaussovskými balíky dobře lokalizovanými okolo bodů $-b$ a $+b$ (obrázek),
        \begin{equation}
            \psi_{\pm}(x)=\frac{1}{\sqrt[4]{\pi\sigma^{2}}}\e^{-\frac{1}{2\sigma^{2}}(x\mp b)^{2}},
        \end{equation}
        kde $\sigma\ll b$ určuje určuje šířku balíku.
        
        \begin{figure}[htbp!]
            \centering
            \includegraphics[width=0.5\linewidth]{Identical.png}
        \end{figure}

        \begin{enumerate}
            \item 
                Určete vlnovou funkci $\psi(x_{1},x_{2})$ systému těchto dvou nerozlišitelných částic a spočítejte její normalizaci.
                Vlnové funkce mohou být symetrické (bosony, fermiony s antisymetrickým spinovým stavem) nebo antisymetrické (fermiony se symetrickým spinovým stavem). 
                Uvažujte oba dva případy výměnné symetrie.
            
            \item Spočítejte střední hodnotu energie systému dvou nerozlišitelných částic
                \begin{equation}
                    E=\matrixelement{\psi}{\operator{H}}{\psi}=\int\psi^{*}(x_{1},x_{2})\operator{H}\psi(x_{1},x_{2})\d x_{1}\d x_{2}.
                \end{equation}

            \item Spočítejte efektivní sílu
                \begin{equation}
                    F\equiv-\frac{\partial E}{\partial b}.
                \end{equation}
                Bude tato síla přitažlivá nebo odpudivá a jak bude záviset na symetrii vlnové funkce?

            \item Určete jednočásticovou hustotu pravděpodobnosti, že nalezneme částici v bodě $x$, a porovnejte ji s hustotou pravděpodobnosti pro rozlišitelné částice.
        \end{enumerate}

\subsection{Helium}
    \begin{enumerate}
        \item
            Napište Hamiltonián pro elektronový obal atomu helia.
            Uvažujte, že atomové jádro je bodová částice mnohem těžší než elektronový obal a zanedbejte jaderné pohyby a spin-orbitální interakci.
            Elektrony jsou nerozlišitelné fermiony se spinem $1/2$.

        \item
            V nultém přiblížení zanedbejte vzájemnou interakci obou elektronů a určete základní stav a první excitovaný stav atomu helia (energie a vlnové funkce).
            V základním stavu bude spinový stav singletní (parahelium) nebo tripletní (ortohelium)?

        \item
            V prvním přiblížení uvažujte vzájemnou interakci obou elektronů jako poruchu.
            Určete opravu k energii, která se spočítá jako střední hodnota interakčního členu z Hamiltoniánu pro systém ve stavu popsaném vlnovými funkcemi z předchozího bodu.
    \end{enumerate}

    \subsection{Pozitronium}
        Pozitronium je vázaný systém elektronu a pozitronu.\footnote{Až na výjimky je každé \emph{-onium} vázaný systém částice a své antičástice.}
        Určete jeho poloměr a energetické spektrum a porovnejte se spektrem atomu vodíku.

        \emph{Poznámka:} Spin elektronu a pozitronu se opět složí na singletní $^{1}\mathrm{S}_{0}$ stav (parapozitronium) a tripletní $^{3}\mathrm{S}_{1}$ stav (ortopozitronium).
        Parapozitronium se rozpadá do sudého počtu fotonů, a má tedy mnohem kratší dobu života, $\tau_{0}\approx0\c12\unit{\mu s}$.
        Ortopozitronium má sice energii základního stavu o zhruba $1\unit{meV}$ výše než parapozitronium, ale rozpadá se do lichého počtu fotonů s dobou života více jak tisíckrát delší, $\tau_{1}\approx140\unit{\mu s}$.

